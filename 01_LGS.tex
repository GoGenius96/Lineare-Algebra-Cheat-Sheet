\documentclass{article}
\usepackage[landscape]{geometry}
\usepackage{url}
\usepackage{multicol}
\usepackage{amsmath}
\usepackage{esint}
\usepackage{amsfonts}
\usepackage{tikz}
\usetikzlibrary{decorations.pathmorphing}
\usepackage{amsmath,amssymb}

\usepackage{colortbl}
\usepackage{xcolor}
\usepackage{mathtools}
\usepackage{amsmath,amssymb}

\usepackage{enumerate}
\usepackage{hyperref}
\makeatletter

\newcommand{\N}{\mathbb{N}}
\newcommand{\Z}{\mathbb{Z}}
\newcommand{\Q}{\mathbb{Q}}
\newcommand{\R}{\mathbb{R}}
\newcommand{\C}{\mathbb{C}}
\newcommand{\K}{\mathbb{K}}
\newcommand{\m}{\cdot}
\newcommand{\vect}[1]{\mathbf{#1}} 
\newenvironment{roweqmat}[1]{\left(\array{@{}#1@{}}}{\endarray\right)}

\definecolor{mycolor}{HTML}{3C8031}

\newcommand*\bigcdot{\mathpalette\bigcdot@{.5}}
\newcommand*\bigcdot@[2]{\mathbin{\vcenter{\hbox{\scalebox{#2}{$\m@th#1\bullet$}}}}}
\makeatother

\title{Lineare Algebra für Statistiker Cheat Sheet}
\usepackage[brazilian]{babel}
\usepackage[utf8]{inputenc}

\advance\topmargin-.8in
\advance\textheight3in
\advance\textwidth3in
\advance\oddsidemargin-1.5in
\advance\evensidemargin-1.5in
\parindent0pt
\parskip2pt
\newcommand{\hr}{\centerline{\rule{3.5in}{1pt}}}
%\colorbox[HTML]{e4e4e4}{\makebox[\textwidth-2\fboxsep][l]{texto}
\begin{document}

\begin{multicols*}{3}

\tikzstyle{mybox} = [draw=black, fill=white, very thick,
    rectangle, rounded corners, inner sep=10pt, inner ysep=10pt]
\tikzstyle{fancytitle} =[fill=black, text=white, font=\bfseries]

%------------ Lineare Gleichung ---------------
\begin{tikzpicture}
\node [mybox] (box){%
    \begin{minipage}{0.3\textwidth}
    Eine \textcolor{red}{lineare Gleichung} in $\R^n$ mit Variablen $x_1, \dots, x_n \in \R$ und Koeffizienten $a_1, \dots, a_n, b \in \R$ hat die Form $a_1 x_1 + \dots + a_n x_n = b.$
    \end{minipage}
};

%------------ Lineare Gleichung Header ---------------------
\node[fancytitle, right=10pt] at (box.north west) {Lineare Gleichung};
\end{tikzpicture}

%------------ LGS ---------------
\begin{tikzpicture}
\node [mybox] (box){%
    \begin{minipage}{0.3\textwidth}
    Ein \textcolor{red}{lineares Gleichungssystem (LGS)} aus $m$ Gleichungen und $n$ Unbekannten $x_1, \dots, x_n$ hat die Form
	\begin{alignat*}{6}
		 & a_{11} x_1  & {}+{} & \dots           & {}+{} & a_{1n} x_n & {}={} & b_1  \\
		 & a_{21}  x_1 & {}+{} & \dots           & {}+{} & a_{2n} x_n & {}={} & b_2  \\
		 &             &       & \quad \; \vdots                                     \\
		 & a_{m1}  x_1 & {}+{} & \dots           & {}+{} & a_{mn} x_n & {}={} & b_m.
	\end{alignat*}
     Wir nennen zwei LGS \textcolor{red}{äquivalent}, wenn sie die gleichen Lösungen besitzen. Ein LGS lässt sich vollständig durch die Koeffizienten beschreiben und kann daher auch in Form einer \textcolor{red}{(erweiterten) Koeffizientenmatrix} dargestellt werden.
     \begin{align*}
    		\begin{roweqmat}{ccc|c}
    			a_{11} & \dots & a_{1n} & b_1 \\
    			\vdots &   & \vdots & \vdots \\
    			a_{m1} & \cdots & a_{mn} & b_m
    		\end{roweqmat}.
    	\end{align*}
    Ohne die letzte Spalte spricht man von einer Koeffizientenmatrix.
    \end{minipage}
};
%------------ LGS Header ---------------------
\node[text=white, fancytitle, right=10pt] at (box.north west) {Lineares Gleichungssystem};
\end{tikzpicture}

%------------ LGS Beispiel ---------------
\begin{tikzpicture}
\node [mybox] (box){%
    \begin{minipage}{0.3\textwidth}
        \begin{minipage}{0.4\textwidth}
    	\begin{alignat*}{6}
    		 x_1 & {}+{}  & 5x_2 & {}+{} & 3x_3 & {}={} & -2 \\
    		 2x_1 & {}+{} & 2x_2 & {}+{} & 4x_3 & ={} & 8 \\
    		 -3x_1 & {}+{} & 2x_2 & {}+{} & 9x_3 & {}={} & -1
    	\end{alignat*}
    \end{minipage}
    \begin{minipage}{0.4\textwidth}
    	\begin{align*}
    		\qquad \longrightarrow
    	\begin{roweqmat}{rrr|r}
    		1 & 5 & 3 & -2 \\
    		2 & 2 & 4 & 8 \\
    		-3 & 2 & 9 & -1
    	\end{roweqmat}
    \end{align*}
    \end{minipage}
    \end{minipage}
};
%------------ LGS Beispiel Header ---------------------
\node[fill = blue, text=white, font=\bfseries, right=10pt] at (box.north west) {LGS Beispiel};
\end{tikzpicture}

%------------ Elementare Zeilenoperationen ---------------
\begin{tikzpicture}
\node [mybox] (box){%
    \begin{minipage}{0.3\textwidth}
    Um das LGS zu vereinfachen, ohne die Lösungsmenge zu ändern, sind folgende \textcolor{red}{elementare Zeilenoperationen} erlaubt:

	\begin{enumerate}[(i)]
		\item Multipliziere eine Zeile mit einer Konstante \\$c \neq 0$.
		\item Tausche zwei Zeilen.
		\item Addiere eine Zeile $c \neq 0$ mal zu einer anderen.
	\end{enumerate}
    \end{minipage}
};
%------------ Elementare Zeilenoperationen Header ---------------------
\node[fill = purple, text=white, font=\bfseries, right=10pt] at (box.north west) {Elementare Zeilenoperationen};
\end{tikzpicture}

%------------ Zeilenstufenform ---------------
\begin{tikzpicture}
\node [mybox] (box){%
    \begin{minipage}{0.3\textwidth}
    Eine Matrix ist in \textcolor{red}{Zeilenstufenform}, wenn:
	\begin{enumerate}[(i)]
		\item Alle \textcolor{red}{Nullzeilen} (Zeilen, in der höchstens der letzte Eintrag ungleich 0 ist) befinden sich am Ende der Matrix.
		\item Die \textcolor{red}{Pivots} (der erste Eintrag $a_{ji}$ einer Zeile ungleich Null: $j_i = \min\{j\colon a_{ij} \neq 0\})$ der anderen  Zeilen erfüllen die Stufenbedingung
		      $$j_{1} < j_{2} < \dots < j_{r}.$$
        Variablen die zu einer Pivot gehören nennen wir \textcolor{red}{gebunden} und die restlichen Variablen nennen wir \textcolor{red}{frei}
	\end{enumerate}

    Eine Matrix ist in \textcolor{red}{reduzierter Zeilenstufenform}, wenn sie die folgenden Bedingungen erfüllt:
	\begin{enumerate}[(i)]
		\item Sie ist in Zeilenstufenform.
		\item Alle Pivoteinträge $a_{i j_i}$ sind gleich 1.
		\item Alle Spalteneinträge über einem Pivot sind gleich 0.
	\end{enumerate}
    \end{minipage}
};
%------------ Zeilenstufenform Header ---------------------
\node[fill = black, text=white, font=\bfseries, right=10pt] at (box.north west) {(reduzierte) Zeilenstufenform};
\end{tikzpicture}

%------------ LGS Zeilenstufenform Beispiel ---------------
\begin{tikzpicture}
\node [mybox] (box){%
    \begin{minipage}{0.3\textwidth}
    Das untere LGS ist in Zeilenstufenform, aber nicht in reduzierter Zeilenstufenform.\\
    \begin{minipage}{0.4\textwidth}
	\begin{alignat*}{4}
		 & x_1 {}-{} & 2x_2 {}+{} & 6x_3 & {}={} & 2  \\
		 &           & 7x_2 {}+{} & 3x_3 & {}={} & 2  \\
		 &           &            & 3x_3 & {}={} & 2.
	\end{alignat*}
\end{minipage}
\begin{minipage}{0.4\textwidth}
	\begin{align*}
		\qquad \longrightarrow
	\begin{roweqmat}{rrr|r}
		1 & -2 & 6 & 2 \\
		0 & 7 & 3 & 2 \\
		0 & 0 & 3 & 2
	\end{roweqmat}
\end{align*}
\end{minipage}
    \end{minipage}
};
%------------ LGS Zeilenstufenform Beispiel Header ---------------------
\node[fill = blue, text=white, font=\bfseries, right=10pt] at (box.north west) {LGS in Zeilenstufenform};
\end{tikzpicture}

%------------ Anzahl Lösungen ---------------
\begin{tikzpicture}
\node [mybox] (box){%
    \begin{minipage}{0.3\textwidth}
    \begin{enumerate}[(i)]
		\item 	Ein LGS hat genau dann keine Lösung, wenn sich durch elementare Zeilenoperationen eine \emph{Nullzeile}
		      $$ \begin{roweqmat}{rrrr|c} 0 & 0 & \cdots & 0 & c \end{roweqmat} $$
		      mit $c \neq 0$ erzeugen lässt.
		\item Ein lösbares LGS hat genau eine Lösung, wenn es keine freien Variablen gibt. Andernfalls gibt es unendlich viele Lösungen.
	\end{enumerate}
    \end{minipage}
};
%------------ Anzahl Lösungen Header ---------------------
\node[fill = purple, text=white, font=\bfseries, right=10pt] at (box.north west) {Anzahl Lösungen};
\end{tikzpicture}

%------------ Elementare Zeilenoperationen Beispiele---------------
\begin{tikzpicture}
\node [mybox] (box){%
    \begin{minipage}{0.3\textwidth}
    \begin{eqnarray*}
		& & \begin{roweqmat}{rrr|c}
			3 & 6 & -6 & 3 \\
			1 & 2 & 0 & 3 \\
			-2 & -1 & 9 & -2
		\end{roweqmat} \\
		&  \stackrel{\frac13 \cdot (1)}{\sim} &
		\begin{roweqmat}{rrr|c}
			1 & 2 & -2 & 1 \\
			1 & 2 & 0 & 3 \\
			-2 & -1 & 9 & -2
		\end{roweqmat}
		\\
		& \stackrel{(2) - (1)}{\sim} &
		\begin{roweqmat}{rrr|c}
			1 & 2 & -2 & 1 \\
			0 & 0 & 2 & 2 \\
			-2 & -1 & 9 & -2
		\end{roweqmat}
		\\
		& \stackrel{(2)\leftrightarrow (3)}{\sim} &
		\begin{roweqmat}{rrr|c}
			1 & 2 & -2 & 1 \\
			-2 & -1 & 9 & -2  \\
			0 & 0 & 2 & 2
		\end{roweqmat}
		\\
		& \stackrel{(2) + 2 \cdot (1)}{\sim}	&
		\begin{roweqmat}{rrr|c}
			1 & 2 & -2 & 1 \\
			0 & 3 & 5 & 0 \\
			0 & 0 & 2 & 2
		\end{roweqmat}.
	\end{eqnarray*}
 Jetzt ist die erweiterte Koeffizientenmatrix in Zeilenstufenform.
    \begin{eqnarray*}
		&  \stackrel{\frac13 \cdot (2)}{\sim} &
		\begin{roweqmat}{rrr|c}
			1 & 2 & -2 & 1 \\
			0 & 1 & \frac53 & 0 \\
			0 & 0 & 2 & 2
		\end{roweqmat}
		\\
		& \stackrel{(1) - 2\cdot(2)}{\sim} &
		\begin{roweqmat}{rrr|c}
			1 & 0 & -\frac{16}{3}  & 1 \\
			0 & 1 & \frac53 & 0 \\
			0 & 0 & 2 & 2
		\end{roweqmat}
		\\
		& \stackrel{\frac12 \cdot (3)}{\sim} &
		\begin{roweqmat}{rrr|c}
			1 & 0 & -\frac{16}{3}  & 1 \\
			0 & 1 & \frac53 & 0 \\
			0 & 0 & 1 & 1
		\end{roweqmat}
		\\
		& \stackrel{(1) + \frac{16}{3} \cdot (3)}{\sim}	&
		\begin{roweqmat}{rrr|c}
			1 & 0 & 0 & \frac{19}{3} \\
			0 & 1 & \frac53 & 0 \\
			0 & 0 & 1 & 1
		\end{roweqmat}
        \\
		& \stackrel{(2) - \frac{5}{3} \cdot (3)}{\sim}	&
		\begin{roweqmat}{rrr|c}
			1 & 0 & 0 & \frac{19}{3} \\
			0 & 1 & 0 & - \frac53 \\
			0 & 0 & 1 & 1
		\end{roweqmat}
	\end{eqnarray*}
    Jetzt ist die erweiterte Koeffizientenmatrix in reduzierter Zeilenstufenform.
    \end{minipage}
};
%------------ Elementare Zeilenoperationen Beispiele Header---------------------
\node[fill = blue, text=white, font=\bfseries, right=10pt] at (box.north west) {Elementare Zeilenoperationen Beispiel};
\end{tikzpicture}

%------------ Gauss-Jordan-Verfahren ---------------
\begin{tikzpicture}
\node [mybox] (box){%
    \begin{minipage}{0.3\textwidth}
    Um die Lösungen von einem LGS zu bestimmen, kann man das Gauss-Jordan-Verfahren benutzen:
    \begin{enumerate}[(i)]
	\item Schreibe das LGS in eine Koeffizientenmatrix.
	\item Bringe Matrix durch elementare Zeilenoperationen in Zeilenstufenform.
	\item Stelle fest, ob das System lösbar ist.
	\item Falls ja, reduziere die Matrix um Lösungen zu bestimmen.
\end{enumerate}
    \end{minipage}
};
%------------ Gauss-Jordan-Verfahren Header ---------------------
\node[fill = mycolor, text=white, font=\bfseries, right=10pt] at (box.north west) {Gauss-Jordan-Verfahren};
\end{tikzpicture}


%------------ Überschrift ---------------
\begin{tikzpicture}
\node [mybox] (box){%
    \begin{minipage}{0.3\textwidth}
    
    \end{minipage}
};
%------------ Überschrift Header ---------------------
\node[fill = black, text=white, font=\bfseries, right=10pt] at (box.north west) {Überschrift};
\end{tikzpicture}




%------------ Überschrift ---------------
\begin{tikzpicture}
\node [mybox] (box){%
    \begin{minipage}{0.3\textwidth}
    
    \end{minipage}
};
%------------ Überschrift Header ---------------------
\node[fill = black, text=white, font=\bfseries, right=10pt] at (box.north west) {Überschrift};
\end{tikzpicture}

\end{multicols*}

\end{document}


